\documentclass[12pt]{article}
\usepackage{graphicx} % Required for inserting images

\title{Lab 3: }
\author{Owen Strong \\
\itshape Prof.: Angela DiFulvio\\
\itshape TA: Kholod Mahmoud \\
\itshape TA: Schaffer Bauer \\
\itshape TA: Justin Jia \\
}
\date{7 February 2026}

\usepackage[
    style=ieee
]{biblatex}
\bibliography{refs.bib}

\usepackage[letterpaper, margin=1in]{geometry}
\newcommand{\figwidth}{0.75\linewidth}

\newcommand{\micCi}{$\mathrm{\mu Ci}$}

\usepackage{placeins} % FloatBarrier
\usepackage{amsmath} % align

\begin{document}

\maketitle
\pagebreak
\tableofcontents
\begin{abstract}

\end{abstract}

\section{Introduction and Background}

\section{Experimental Procedure}\label{sec:methods}
\section{Experimental Results}\label{sec:results}

%%%% Experiment 1

% Plot the counting curve for the GM counter. Report the plateau slope.
For a range of voltage bias settings, we measured detector pulses successfully recorded over 30-second periods with a $0.1$~.
These counts are compared to the biases used in Figure~\ref{fig:exp1-bias-v-counts}.
As expected, the measured counts begin to plateau around 700V, before increasing rapidly with increased voltage after 950V.
This plateau region had a slope of approximately 0.680 counts for every additional volt of increased bias.\\

\begin{figure}[!h]
    \centering
    \includegraphics[width=\figwidth]{figs/Exp1_counts-v-volt.png}
    \caption{Radiation counts for a 0.1\micCi measured by the GM detector over 30 second intervals with the Power Supply set to various voltages.
    A plateau region is observed, with a slope of 0.680 counts per volt of bias increase.}
    \label{fig:exp1-bias-v-counts}
\end{figure}

Based on this voltage behavior, we selected an appropriate operating voltage for further experiments.
Two factors motivated our choice, resulting in two different voltage choices for experiments 2 and 3.
Given potential uncertainty in voltage supply, a stable count rate was desired with respect to changes in voltage.
Following this motivation, we chose a voltage of 850V, a point closer to the center of the plateau on Figure~\ref{fig:exp1-bias-v-counts}, where we visually judged the slope to be the least.
This voltage was used for experiment 2.\\

Our second motivation was to minimize the effect of dead time on measurements.
Optimizing for this motivation, an appropriate voltage might be one at the start of the plateau (in our case 700V), to achieve a high count rate without drastically raising the rate of count loss due to dead time.\\

Between experiment 2 and experiment 3, upon feedback from the professor we decided to instead prioritize minimal dead time loss, and changed our voltage supply bias to 700V. 
Experiment 3 was conducted at this lower voltage.\\

% Experiment 2
We measured 
% What the counts were, associated rates and uncertainty were
% Set up system of equations for dead time of detector usning the two sources that we assume are equal using their gross count rates
% Solve for associated uncertainty Do we show formula for that?
% Show and explain how we recorded the dead time


% Experiment 3
% Yada yada were measured for over 45 minutes
% plotted in figure deeedum
% calculate half life, showing regression
% Explain how curve doesn't look that curvy because we only record for 45 minutes, compared to the half life of about an hour so any noise keeps it from looking right.
\section{Discussion}\label{sec:disc}
\section{Conclusions}\label{sec:conc}

% \bibliographystyle{sty/ieeetran}
\printbibliography
\end{document}

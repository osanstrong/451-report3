\documentclass[12pt]{article}
\usepackage{graphicx} % Required for inserting images

\title{Lab 3: Proportional Counters and Geiger Muller (GM) Counters}
\author{Owen Strong \\
\itshape Prof.: Angela DiFulvio\\
\itshape TA: Kholod Mahmoud \\
\itshape TA: Shaffer Bauer \\
\itshape TA: Justin Jia \\
}
\date{7 February 2026}

\usepackage[
    style=ieee
]{biblatex}
\bibliography{refs.bib}

\usepackage[letterpaper, margin=1in]{geometry}
\newcommand{\figwidth}{0.75\linewidth}

\newcommand{\micCi}{$\mathrm{\mu Ci}$}
\newcommand{\InMeta}{$\mathrm{^{116m}In}$}

\usepackage{placeins} % FloatBarrier
\usepackage{amsmath} % align

\begin{document}

\maketitle
\pagebreak
\tableofcontents
\pagebreak
\begin{abstract}

\end{abstract}

\section{Introduction and Background}


\subsection{Half Life and activity}
Every nucleus of a specific radioactive species, has a definite probability within a unit of time of decaying.\cite{npre_lab_3_manual_2025}
As a result, the rate of decay of a given specimen of that isotope at a given instant is directly proportional to the total number of isotope nuclei present:
\begin{equation}
    \frac{dN}{dt}=-\lambda N
\end{equation}
where $N$ corresponds to the number of isotopes present at a time $t$.
The constant $\lambda$ is a ``decay constant'' specific to the isotope, and is a measure of decay probability per unit time.
Separating and integrating this relation yields an exponential description of isotope quantity within that specimen:
\begin{equation}\label{eq:decay}
    N(t)=N_0e^{-\lambda t}
\end{equation}
where $N_0$ corresponds to the quantity at time $t=0$.\\

A more common way than $\lambda$ of describing the decay of a given radioactive species is the half-life, denoted $t_{1/2}$.
The half life represents the amount of time after which half of the original sample will have decayed, and may be given by the formula:
\begin{equation}\label{eq:half-life}
    t_{1/2} = \frac{\ln(2)}{\lambda}
\end{equation}

This quantity is readily available for most radioactive species in data collections such as those published by the Korean Atomic Energy Research Institute.\cite{noauthor_atomkaerirekrcgi-binnuclidenucin116_nodate}

\section{Experimental Procedure}\label{sec:methods}
\section{Experimental Results}\label{sec:results}
After performing the above procedures, we observed the following results:

%%%% Experiment 1
\subsection{Experiment 1: Measurement of the Counting Curve}
% Plot the counting curve for the GM counter. Report the plateau slope.
For a range of voltage bias settings, we measured detector pulses successfully recorded over 30-second periods with a $0.1$~.
These counts are compared to the biases used in Figure~\ref{fig:exp1-bias-v-counts}.
As expected, the measured counts begin to plateau around 700V, before increasing rapidly with increased voltage after 950V.
This plateau region had a slope of approximately 0.680 counts for every additional volt of increased bias.\\

\begin{figure}[!h]
    \centering
    \includegraphics[width=\figwidth]{figs/Exp1_counts-v-volt.png}
    \caption{Radiation counts for a 0.1\micCi measured by the GM detector over 30 second intervals with the Power Supply set to various voltages.
    A plateau region is observed, with a slope of 0.680 counts per volt of bias increase.}
    \label{fig:exp1-bias-v-counts}
\end{figure}

Based on this voltage behavior, we selected an appropriate operating voltage for further experiments.
Two factors motivated our choice, resulting in two different voltage choices for experiments 2 and 3.
Given potential uncertainty in voltage supply, a stable count rate was desired with respect to changes in voltage.
Following this motivation, we chose a voltage of 850V, a point closer to the center of the plateau on Figure~\ref{fig:exp1-bias-v-counts}, where we visually judged the slope to be the least.
This voltage was used for experiment 2.\\

Our second motivation was to minimize the effect of dead time on measurements.
Optimizing for this motivation, an appropriate voltage might be one at the start of the plateau (in our case 700V), to achieve a high count rate without drastically raising the rate of count loss due to dead time.\\

Between experiment 2 and experiment 3, upon feedback from the professor we decided to instead prioritize minimal dead time loss, and changed our voltage supply bias to 700V. 
Experiment 3 was conducted at this lower voltage.\\


\subsection{Dead Time Measurement}
% Experiment 2
The pulse output behavior of the GM Detector was measured and compared over prolonged exposure.
Using the Oscilloscope set to visualize signal traces, the combined trace of these signals compared in time to the signal before them can be seen in Figure~\ref{fig:exp2-osc-dead}.\\


\begin{figure}
    \centering
    \includegraphics[width=\figwidth]{figs/Exp2_oscDeadTime.jpg}
    \caption{Traces (faded yellow) of signals (bright yellow) from the GM Detector while exposed to the 30~\micCi $\beta$ source.
    Visualized using the Oscilloscope, with traces set to visible. 
    Note the gap after the initial signal peak, where following pulses did not leave a trace, followed by a ``recovery'' period in which following pulses were substantially attenuated.}
    \label{fig:exp2-osc-dead}
\end{figure}

Using the measure function of the Oscilloscope, we measured the dead time, where no followup pulse could be detected, to be $562\pm2~\mathrm{\mu s}$.
Measuring from the initial peak instead to where traces reached 95\% of their base amplitude ($12.6875\pm0.0625$~V), we measured a 95\% recovery time of $1384\pm2~\mathrm{\mu s}$.\\

We measured three 60-second trials each of GM Detector counts while exposed to 1\micCi~source 1, 1\micCi~source 2 and 1\micCi~source 2 simultaneously, then 1\micCi~source 2 alone, for a total of nine trials.
For each set of 60-second counts $C_1, C_2, C_3$, we calculated a base count rate:
\begin{equation}\label{eq:count-rate}
    m_0=\frac{C_1+C_2+C_3}{180~\mathrm{seconds}}
\end{equation}
And, modeling counts as a Poisson distribution, an uncertainty for this rate:
\begin{equation}\label{eq:uncertainty}
    \sigma_0=\frac{\sqrt{C_1+C_2+C_3~\mathrm{counts~within~180~seconds}}}{\mathrm{180~seconds}}
\end{equation}
Then, we accounted for the background rate.
For the identical measurement over 5 minutes (300 seconds), we measured 95 counts, yielding as a Poisson distribution an uncertainty of $\sqrt{95}$.
This corresponds to a background count rate and uncertainty of:
\begin{equation}
    m_b\pm\sigma_b=95~\mathrm{counts}\times\left(300~\mathrm{s}\right)^{-1}\pm\frac{\sqrt{95}}{300}=0.317\pm0.032~\mathrm{s}^-1
\end{equation}

Thus, to calculate the net count rate and uncertainty for each source:
\begin{equation}
    m=m_0-m_b
\end{equation}
\begin{equation}\label{eq:new-uncertainty}
    \sigma=\sqrt{\sigma_0 + \sigma_b}
\end{equation}

From source 1 alone, we measured 60-second counts of 131, 118, and 124, yielding with Equations~\ref{eq:count-rate} through \ref{eq:new-uncertainty}, a count rate $m_1=1.756\pm0.112~\mathrm{s^{-1}}$.\\

For sources 1 and 2 combined, we measured 60-second counts of 227, 216, and 226, yielding a count rate $m_{12}=3.400\pm0.147\mathrm{s^{-1}}$.\\

Finally, for source 2 alone, we measured 60-second counts of 120, 137, and 106, yielding a count rate $m_2=1.700\pm0.111\mathrm{s^{-1}}$.\\

Using these rate measurements, we were able to estimate the dead time of the detector using Equation~\ref{eq:dead-time}:
\begin{equation}\label{eq:dead-time}
    \tau(m_1,m_{12},m_2)=\frac{1-\sqrt{1-\frac{m_{12}\left(m_1+m_2-m_{12}\right)}{m_1m_2}}}{m_{12}}
\end{equation}

Using the derivatives of this function with respect to each uncertain count rate:


% (which may be found in Appendix~ at \url{https://www.desmos.com/calculator/6eplmvfm3o}):
\begin{equation}\label{eq:t-m1}
    \tau_{m_1}=-\frac{\frac{1}{m_{12}}\left(\frac{-m_{12}m_{1}+m_{12}\left(m_{1}+m_{2}-m_{12}\right)}{m_{1}^{2}m_{2}}\right)}{2\sqrt{1-\frac{m_{12}\left(m_{1}+m_{2}-m_{12}\right)}{m_{1}m_{2}}}}
\end{equation}
\begin{equation}
    \tau_{m_2}=-\frac{\frac{1}{m_{12}}\left(\frac{-m_{12}m_{2}+m_{12}\left(m_{2}+m_{1}-m_{12}\right)}{m_{2}^{2}m_{1}}\right)}{2\sqrt{1-\frac{m_{12}\left(m_{2}+m_{1}-m_{12}\right)}{m_{2}m_{1}}}}
\end{equation}
\begin{equation}
    \tau_{m_{12}}=\frac{\left(-m_{12}\frac{\left(\frac{2m_{12}-\left(m_{1}+m_{2}\right)}{m_{1}m_{2}}\right)}{2\sqrt{1-\frac{m_{12}\left(m_{1}+m_{2}-m_{12}\right)}{m_{1}m_{2}}}}+\sqrt{1-\frac{m_{12}\left(m_{1}+m_{2}-m_{12}\right)}{m_{1}m_{2}}}-1\right)}{m_{12}^{2}}
\end{equation}

we calculated the uncertainty of the dead time $\tau$ as a function of $m_1$, $m_{12}$, and $m_2$:
\begin{equation}\label{eq:t-unc}
    \sigma_\tau=\sqrt{\tau_{m_1}^2\sigma_1^2 + \tau_{m_{12}}^2\sigma_{12}^2 + \tau_{m_2}^2\sigma_2^2}
\end{equation}

Using Equations~\ref{eq:dead-time} through \ref{eq:t-unc}, we calculated a dead time of $9.460\pm36.700$~ms.
Demonstrations of our full numerical calculations may be found at \url{https://www.desmos.com/calculator/6eplmvfm3o}.
% What the counts were, associated rates and uncertainty were
% Set up system of equations for dead time of detector usning the two sources that we assume are equal using their gross count rates
% Solve for associated uncertainty Do we show formula for that?
% Show and explain how we recorded the dead time


% Experiment 3
\FloatBarrier\subsection{Experiment 3: Rate of Radioactive Decay}

As measured by GM detector counts within a span of 60 seconds, we measured activity of the \InMeta foil sample every 90 seconds over a period of approximately 49 minutes.
The relative activity of the sample through these counts over time can be observed in Figure~\ref{fig:exp2-116m}.\\

\begin{figure}[!htb]
    \centering
    \includegraphics[width=\figwidth]{figs/Exp3_in116-activity.png}
    \caption{Relative activity (as measured by GM detector counts over 60 seconds) of the \InMeta source every 90 seconds over a period spanning 49.5 minutes.}
    \label{fig:exp3-in116m}
\end{figure}

As seen in Equation~\ref{eq:decay}, we expect the remaining quantity of an isotope, and thus its activity, to follow an exponential decay with respect to time.
If we take the natural logarithm of the expected activity:
\begin{equation}\label{eq:log-decay}
    \ln(N)=-\lambda t + \ln(N_0)
\end{equation}
we find that we expect a linear relationship, where slope corresponds to the isotope's decay constant $\lambda$.
Using Equation~\ref{eq:log-decay}, we fit a linear relationship between the logarithm of a measurement and the elapsed time before that measurement.
We found a reasonable relationship ($R^2=0.806$), with a slope $-\lambda=-0.0109$.
Using Equation~\ref{eq:half-life}, we calculated an estimated half life of $63.606$~minutes.
% Yada yada were measured for over 45 minutes
% plotted in figure deeedum
% calculate half life, showing regression
% Explain how curve doesn't look that curvy because we only record for 45 minutes, compared to the half life of about an hour so any noise keeps it from looking right.
\FloatBarrier
\section{Discussion}\label{sec:disc}
In experiment~1, while we find a very suitable plateau in the change of count rate with respect to GM detector voltage, there was some dispute in how to use this to calculate a suitable operating voltage.\\

One motivation is to minimize the possible effect of variation in voltage supply.
Under this motivation, a suitable voltage would be more towards the middle of the plateau, around $850$~V where we performed experiment~2.
However, a second motivation would be to minimize \emph{any} effect of dead time on measurements for the same strength as the 1.0~\micCi source used for experiment 1, while allowing for observation of the effect with stronger sources.
This way, a clear distinction could be made between this measurement and a stronger one.
This would prompt choice of an operating voltage closer to the start of the peak, around $700$~V, as was used to perform experiment~3.\\

It is very possible that our results from experiment~2 demonstrate why a choice of 700~V would be superior.
While we were able to estimate the dead time of our detector using the measurements of sources~1 alone, 1 and 2, and 2 alone (at 9.460~ms), the uncertainty of this estimate ($\mathrm{\pm 36.700}~ms$) far exceeded the magnitude of that estimate.
Additionally, even before calculating the uncertainty of this estimate, the two source estimate of 9.460~ms was a different order of magnitude from the dead time observed using the oscilloscope trace in Figure~\ref{fig:exp2-osc-dead}, which we estimated to be $0.562\pm0.002$~ms.
Even judging conservatively, assuming that pulses would not be properly counted until 95\% recovery, this would yield a recovery time from the same trace of Figure~\ref{fig:exp2-osc-dead} of $1.384\pm0.002$~ms, still far less than that calculated by the two source method.\\

It is very possible, although not certain, that our selection of 850~V could have negatively impacted our estimate. 
As the purpose of comparing different source strengths is to compare the effect of dead time on count rates (with the formula simply being an abstraction of this to take three measurements and yield a dead time), an ideal measurement may have been taken at the start of the plateau for the strength of just one source, around 700~V in Figure~\ref{fig:exp1-bias-v-counts}.
If measurements for a single source fall around the start of that plateau, then we would expect dead time to have minimal effect while essentially still maximizing the comparative effect of dead time on the measurement of two sources simultaneously.\\

In experiment 3, there was a stronger than expected noise in count rates which impacted our ability to estimate half life, as is most clear when viewing the measurements over time in Figure~\ref{fig:exp3-in116m}.
Our half-life estimate of 63.606 minutes is comparable to the literature value of 54.29~minutes\cite{noauthor_atomkaerirekrcgi-binnuclidenucin116_nodate}, but not comfortably close, and the regression of our data we used to estimate this value only had a correlation of $R^2=0.806$.
Another factor that definitely compounded with the noise present was the short time period.
Due to time constraints from being required to manually perform measurements during class time, we were only able to perform measurements over a period of 49 minutes, not even a full half life of the material.
Combined with noise, the exponential relationship itself is not even visually obvious without context that the data came from radioactive decay.
A longer measurement period, perhaps by automating the measurement procedure, would likely have gone a long way in providing certainty on the exponential relationship and decay rate.

%  $562\pm2~\mathrm{\mu s}$.
% Measuring from the initial peak instead to where traces reached 95\% of their base amplitude ($12.6875\pm0.0625$~V), we measured a 95\% recovery time of $1384\pm2~\mathrm{\mu s}$.\\
\section{Conclusions}\label{sec:conc}

\appendix
\section{Equipment Details}
In order to better replicate experiments and identify issues, the details of each equipment piece are recorded in Table~\ref{tab:equip}

\newcommand{\na}{$N/A$}
\begin{table}
    \centering
    \caption{Details of the equipment used in this laboratory.
    Where a detail could not be identified for a given device, it is replaced with ``\na''.}\label{tab:equip}
    \begin{tabular}{c|ccc}
        Item & Serial Number & Manufacturer & Address \\

    \end{tabular}

\end{table}

% \bibliographystyle{sty/ieeetran}
\printbibliography
\end{document}

\documentclass[12pt]{article}
\usepackage{graphicx} % Required for inserting images

\title{Lab 3: }
\author{Owen Strong \\
\itshape Prof.: Angela DiFulvio\\
\itshape TA: Kholod Mahmoud \\
\itshape TA: Shaffer Bauer \\
\itshape TA: Justin Jia \\
}
\date{7 February 2026}

\usepackage[
    style=ieee
]{biblatex}
\bibliography{refs.bib}

\usepackage[letterpaper, margin=1in]{geometry}
\newcommand{\figwidth}{0.75\linewidth}

\newcommand{\micCi}{$\mathrm{\mu Ci}$}

\usepackage{placeins} % FloatBarrier
\usepackage{amsmath} % align

\begin{document}

\maketitle
\pagebreak
\tableofcontents
\pagebreak
\begin{abstract}

\end{abstract}

\section{Introduction and Background}

\section{Experimental Procedure}\label{sec:methods}
\section{Experimental Results}\label{sec:results}

%%%% Experiment 1

% Plot the counting curve for the GM counter. Report the plateau slope.
For a range of voltage bias settings, we measured detector pulses successfully recorded over 30-second periods with a $0.1$~.
These counts are compared to the biases used in Figure~\ref{fig:exp1-bias-v-counts}.
As expected, the measured counts begin to plateau around 700V, before increasing rapidly with increased voltage after 950V.
This plateau region had a slope of approximately 0.680 counts for every additional volt of increased bias.\\

\begin{figure}[!h]
    \centering
    \includegraphics[width=\figwidth]{figs/Exp1_counts-v-volt.png}
    \caption{Radiation counts for a 0.1\micCi measured by the GM detector over 30 second intervals with the Power Supply set to various voltages.
    A plateau region is observed, with a slope of 0.680 counts per volt of bias increase.}
    \label{fig:exp1-bias-v-counts}
\end{figure}

Based on this voltage behavior, we selected an appropriate operating voltage for further experiments.
Two factors motivated our choice, resulting in two different voltage choices for experiments 2 and 3.
Given potential uncertainty in voltage supply, a stable count rate was desired with respect to changes in voltage.
Following this motivation, we chose a voltage of 850V, a point closer to the center of the plateau on Figure~\ref{fig:exp1-bias-v-counts}, where we visually judged the slope to be the least.
This voltage was used for experiment 2.\\

Our second motivation was to minimize the effect of dead time on measurements.
Optimizing for this motivation, an appropriate voltage might be one at the start of the plateau (in our case 700V), to achieve a high count rate without drastically raising the rate of count loss due to dead time.\\

Between experiment 2 and experiment 3, upon feedback from the professor we decided to instead prioritize minimal dead time loss, and changed our voltage supply bias to 700V. 
Experiment 3 was conducted at this lower voltage.\\

% Experiment 2
The pulse output behavior of the GM Detector was measured and compared over prolonged exposure.
Using the Oscilloscope set to visualize signal traces, the combined trace of these signals compared in time to the signal before them can be seen in Figure~\ref{fig:exp2-osc-dead}.\\


\begin{figure}
    \centering
    \includegraphics[width=\figwidth]{figs/Exp2_oscDeadTime.jpg}
    \caption{Traces (faded yellow) of signals (bright yellow) from the GM Detector while exposed to the 30~\micCi $\beta$ source.
    Visualized using the Oscilloscope, with traces set to visible. 
    Note the gap after the initial signal peak, where following pulses did not leave a trace, followed by a ``recovery'' period in which following pulses were substantially attenuated.}
    \label{fig:exp2-osc-dead}
\end{figure}

Using the measure function of the Oscilloscope, we measured the dead time, where no followup pulse could be detected, to be $562\pm2~\mathrm{\mu s}$.
Measuring from the initial peak instead to where traces reached 95\% of their base amplitude ($12.6875\pm0.0625$~V), we measured a 95\% recovery time of $1,384\pm2~\mathrm{\mu s}$.\\

We measured three 60-second trials each of GM Detector counts while exposed to 1\micCi~source 1, 1\micCi~source 2 and 1\micCi~source 2 simultaneously, then 1\micCi~source 2 alone, for a total of nine trials.
For each set of 60-second counts $C_1, C_2, C_3$, we calculated a base count rate:
\begin{equation}\label{eq:count-rate}
    m_0=\frac{C_1+C_2+C_3}{180~\mathrm{seconds}}
\end{equation}
And, modeling counts as a Poisson distribution, an uncertainty for this rate:
\begin{equation}\label{eq:uncertainty}
    \sigma_0=\frac{\sqrt{C_1+C_2+C_3~\mathrm{counts~within~180~seconds}}}{\mathrm{180~seconds}}
\end{equation}
Then, we accounted for the background rate.
For the identical measurement over 5 minutes (300 seconds), we measured 95 counts, yielding as a Poisson distribution an uncertainty of $\sqrt{95}$.
This corresponds to a background count rate and uncertainty of:
\begin{equation}
    m_b\pm\sigma_b=95~\mathrm{counts}\times\left(300~\mathrm{s}\right)^{-1}\pm\frac{\sqrt{95}}{300}=0.317\pm0.032~\mathrm{s}^-1
\end{equation}

Thus, to calculate the net count rate and uncertainty for each source:
\begin{equation}
    m=m_0-m_b
\end{equation}
\begin{equation}\label{eq:new-uncertainty}
    \sigma=\sqrt{\sigma_0 + \sigma_b}
\end{equation}

From source 1 alone, we measured 60-second counts of 131, 118, and 124, yielding with Equations~\ref{eq:count-rate} through \ref{eq:new-uncertainty}, a count rate $m_1=1.756\pm0.112~\mathrm{s^{-1}}$.\\

For sources 1 and 2 combined, we measured 60-second counts of 227, 216, and 226, yielding a count rate $m_{12}=3.400\pm0.147\mathrm{s^{-1}}$.\\

Finally, for source 2 alone, we measured 60-second counts of 120, 137, and 106, yielding a count rate $m_2=1.700\pm0.111\mathrm{s^{-1}}$.\\

Using these rate measurements, we were able to estimate the dead time of the detector using Equation~\ref{eq:dead-time}:
\begin{equation}\label{eq:dead-time}
    \tau(m_1,m_{12},m_2)=\frac{1-\sqrt{1-\frac{m_{12}\left(m_1+m_2-m_{12}\right)}{m_1m_2}}}{m_{12}}
\end{equation}

Using the derivatives of this function with respect to each uncertain count rate:


% (which may be found in Appendix~ at \url{https://www.desmos.com/calculator/6eplmvfm3o}):
\begin{equation}\label{eq:t-m1}
    \tau_{m_1}=-\frac{\frac{1}{m_{12}}\left(\frac{-m_{12}m_{1}+m_{12}\left(m_{1}+m_{2}-m_{12}\right)}{m_{1}^{2}m_{2}}\right)}{2\sqrt{1-\frac{m_{12}\left(m_{1}+m_{2}-m_{12}\right)}{m_{1}m_{2}}}}
\end{equation}
\begin{equation}
    \tau_{m_2}=-\frac{\frac{1}{m_{12}}\left(\frac{-m_{12}m_{2}+m_{12}\left(m_{2}+m_{1}-m_{12}\right)}{m_{2}^{2}m_{1}}\right)}{2\sqrt{1-\frac{m_{12}\left(m_{2}+m_{1}-m_{12}\right)}{m_{2}m_{1}}}}
\end{equation}
\begin{equation}
    \tau_{m_{12}}=\frac{\left(-m_{12}\frac{\left(\frac{2m_{12}-\left(m_{1}+m_{2}\right)}{m_{1}m_{2}}\right)}{2\sqrt{1-\frac{m_{12}\left(m_{1}+m_{2}-m_{12}\right)}{m_{1}m_{2}}}}+\sqrt{1-\frac{m_{12}\left(m_{1}+m_{2}-m_{12}\right)}{m_{1}m_{2}}}-1\right)}{m_{12}^{2}}
\end{equation}

we calculated the uncertainty of the dead time $\tau$ as a function of $m_1$, $m_{12}$, and $m_2$:
\begin{equation}\label{eq:t-unc}
    \sigma_\tau=\sqrt{\tau_{m_1}^2\sigma_1^2 + \tau_{m_{12}}^2\sigma_{12}^2 + \tau_{m_2}^2\sigma_2^2}
\end{equation}

Using Equations~\ref{eq:dead-time} through \ref{eq:t-unc}, we calculated a dead time of $9.460\pm36.700$~ms.
Our full numerical calculations are demonstrated at \url{https://www.desmos.com/calculator/6eplmvfm3o}.
% What the counts were, associated rates and uncertainty were
% Set up system of equations for dead time of detector usning the two sources that we assume are equal using their gross count rates
% Solve for associated uncertainty Do we show formula for that?
% Show and explain how we recorded the dead time


% Experiment 3
% Yada yada were measured for over 45 minutes
% plotted in figure deeedum
% calculate half life, showing regression
% Explain how curve doesn't look that curvy because we only record for 45 minutes, compared to the half life of about an hour so any noise keeps it from looking right.
\FloatBarrier
\section{Discussion}\label{sec:disc}
\section{Conclusions}\label{sec:conc}

\appendix
\section{Equipment Details}
In order to better replicate experiments and identify issues, the details of each equipment piece are recorded in Table~\ref{}

\newcommand{\na}{$N/A$}
\begin{table}
    \centering
    \caption{Details of the equipment used in this laboratory.
    Where a detail could not be identified for a given device, it is replaced with ``\na''.}
    \begin{tabular}{c|ccc}
        Item & Serial Number & Manufacturer & Address \\

    \end{tabular}
\end{table}

% \bibliographystyle{sty/ieeetran}
\printbibliography
\end{document}
